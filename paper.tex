\documentclass[sigplan, review]{acmart}\settopmatter{printfolios=true}
%% For final camera-ready submission
% \documentclass[acmsmall]{acmart}\settopmatter{}


\usepackage{
  , coqdoc
  , todonotes
  , csquotes
  , balance
  , lettrine   % use a large font for first letter
  % , MorrisIn    % font for starting letter of chapter
  % , EileenBl    % font for starting letter of chapter
  , Zallman      % font for starting letter of chapter
}

\usepackage[utf8]{inputenc}

% \renewcommand{\LettrineFontHook}{\MorrisInfamily}
% \renewcommand{\LettrineFontHook}{\EileenBlfamily}
\renewcommand{\LettrineFontHook}{\Zallmanfamily}
\LettrineTextFont{\itshape}
\setcounter{DefaultLines}{3}

\makeatletter
\renewcommand{\verbatim@font}{\ttfamily\small}
\makeatother

\renewcommand{\thesection}{Chapter~\Roman{section}}

% This is a acm-specific command to change the name of figures
\captionsetup[figure]{name={Scroll}}
\renewcommand{\thefigure}{\Roman{figure}}

% Change style of references
\def\nospace~{}
\renewcommand{\sectionautorefname}{\nospace}
\renewcommand{\figureautorefname}{Scroll}

\newcommand{\jump}[1]{\vspace{1ex}\noindent\textit{#1}\vspace{1ex}}
\newenvironment{spec}{\small}{\normalsize}


\renewenvironment{coqdoccomment}{\footnotesize\tt(*}{*)\normalsize}
\renewenvironment{coqdoccode}{\small}{\normalsize}

\bibpunct{(}{)}{;}{a}{},
\let\cite=\citeauthor

%% don't color citations/links green
\hypersetup{
    colorlinks=false,
    pdfborder={0 0 0},
}

\makeatletter\if@ACM@journal\makeatother
%% Journal information (used by PACMPL format)
%% Supplied to authors by publisher for camera-ready submission
\acmJournal{PACMPL}
\acmVolume{1}
\acmNumber{1}
\acmArticle{1}
\acmYear{2018}
\acmMonth{1}
\acmDOI{10.1145/nnnnnnn.nnnnnnn}
\startPage{1}
\else\makeatother
%% Conference information (used by SIGPLAN proceedings format)
%% Supplied to authors by publisher for camera-ready submission
\acmConference[Haskell'18]{ACM SIGPLAN Haskell Symposium 2018}{September 27--28, 2018}{St. Louis, Missouri, USA}
\acmYear{2018}
\acmISBN{978-x-xxxx-xxxx-x/YY/MM}
\acmDOI{10.1145/nnnnnnn.nnnnnnn}
\startPage{1}
\fi


%% Copyright information
%% Supplied to authors (based on authors' rights management selection;
%% see authors.acm.org) by publisher for camera-ready submission
\setcopyright{none}             %% For review submission
%\setcopyright{acmcopyright}
%\setcopyright{acmlicensed}
%\setcopyright{rightsretained}
%\copyrightyear{2017}           %% If different from \acmYear


%% Bibliography style
\bibliographystyle{ACM-Reference-Format}
%% Citation style
%% Note: author/year citations are required for papers published as an
%% issue of PACMPL.
\citestyle{acmauthoryear}   %% For author/year citations


\begin{document}

%% Title information
\title{One Monad to Prove Them All (Functional Pearl)}
% \subtitle{Free Monads, Containers, and Monad-Generic Proofs}
                                        %% [Short Title] is optional;
                                        %% when present, will be used in
                                        %% header instead of Full Title.
%\titlenote{with title note}             %% \titlenote is optional;
                                        %% can be repeated if necessary;
                                        %% contents suppressed with 'anonymous'
%\subtitle{Subtitle}                     %% \subtitle is optional
%\subtitlenote{with subtitle note}       %% \subtitlenote is optional;
                                        %% can be repeated if necessary;
                                        %% contents suppressed with 'anonymous'


%% Author information
%% Contents and number of authors suppressed with 'anonymous'.
%% Each author should be introduced by \author, followed by
%% \authornote (optional), \orcid (optional), \affiliation, and
%% \email.
%% An author may have multiple affiliations and/or emails; repeat the
%% appropriate command.
%% Many elements are not rendered, but should be provided for metadata
%% extraction tools.

%% Author with single affiliation.
\author{Jan Christiansen}
% \authornote{with author1 note}          %% \authornote is optional;
                                        %% can be repeated if necessary
% \orcid{nnnn-nnnn-nnnn-nnnn}             %% \orcid is optional
\affiliation{
  % \position{Position1}
  % \department{Fachbereich 3: Information und Kommunikation}              %% \department is recommended
  \institution{Flensburg University of Applied Sciences}            %% \institution is required
  % \streetaddress{Kanzleistraße 91-93}
  % \city{Flensburg}
  % \state{State1}
  % \postcode{24943}
  % \country{Germany}
}
\email{jan.christiansen@hs-flensburg.de}          %% \email is recommended

%% Author with two affiliations and emails.
\author{Sandra Dylus}
% \authornote{with author2 note}          %% \authornote is optional;
                                        %% can be repeated if necessary
% \orcid{nnnn-nnnn-nnnn-nnnn}             %% \orcid is optional
\affiliation{
  % \position{Position2b}
  % \department{Programmiersprachen und Übersetzerkonstruktion}             %% \department is recommended
  \institution{University of Kiel}           %% \institution is required
  % \streetaddress{Christian-Albrechts-Platz 4}
  % \city{Kiel}
  % \state{State2b}
  % \postcode{24118}
  % \country{Germany}
}
\email{sad@informatik.uni-kiel.de}         %% \email is recommended

%% Author with two affiliations and emails.
\author{Finn Teegen}
% \authornote{with author2 note}          %% \authornote is optional;
                                        %% can be repeated if necessary
% \orcid{nnnn-nnnn-nnnn-nnnn}             %% \orcid is optional
\affiliation{
  % \position{Position2b}
  % \department{Programmiersprachen und Übersetzerkonstruktion}             %% \department is recommended
  \institution{University of Kiel}           %% \institution is required
  % \streetaddress{Christian-Albrechts-Platz 4}
  % \city{Kiel}
  % \state{State2b}
  % \postcode{24118}
  % \country{Germany}
}
\email{fte@informatik.uni-kiel.de}         %% \email is recommended


%% Paper note
%% The \thanks command may be used to create a "paper note" ---
%% similar to a title note or an author note, but not explicitly
%% associated with a particular element.  It will appear immediately
%% above the permission/copyright statement.
% \thanks{with paper note}                %% \thanks is optional
                                        %% can be repeated if necesary
                                        %% contents suppressed with 'anonymous'


%% Abstract
%% Note: \begin{abstract}...\end{abstract} environment must come
%% before \maketitle command
\begin{abstract}
One Monad to Prove Them All is a modern fairy tale about curiosity and perseverance, two important properties of a successful PhD student.
We follow the PhD student Mona on her adventurous journey of proving properties about Haskell programs in the proof assistant Coq.
In quest for a solution Mona goes on an educating journey through the land of functional programming.
We follow her on her journey when she learns about concepts like free
monads and containers as well as basics and restrictions of proof
assistants like Coq.
These concepts are well-known individually, but their interplay gives rise to a solution for Mona's problem that has not been presented before.
Mona's final solution does not only work for a specific monad instance but even allows her to prove monad-generic properties.
Instead of proving properties over and over again for specific monad instances she is able to prove properties that hold for all monads.

If you are a citizen of the land of functional programming or are at least familiar with its customs, had a journey that involved reasoning about functional programs of your own, or are just a curious soul looking for the next story about monads and proofs, then this tale is for you.
\end{abstract}


%% 2012 ACM Computing Classification System (CSS) concepts
%% Generate at 'http://dl.acm.org/ccs/ccs.cfm'.
\begin{CCSXML}
<ccs2012>
<concept>
<concept_id>10003752.10010124.10010138.10010142</concept_id>
<concept_desc>Theory of computation~Program verification</concept_desc>
<concept_significance>500</concept_significance>
</concept>
<concept>
<concept_id>10011007.10011006.10011008.10011009.10011012</concept_id>
<concept_desc>Software and its engineering~Functional languages</concept_desc>
<concept_significance>500</concept_significance>
</concept>
</ccs2012>
\end{CCSXML}

\ccsdesc[500]{Theory of computation~Program verification}
\ccsdesc[500]{Software and its engineering~Functional languages}
%% End of generated code


%% Keywords
%% comma separated list
\keywords{Haskell, monads, free monad, containers, verification, non-determinism, fairy tale}  %% \keywords is optional


%% \maketitle
%% Note: \maketitle command must come after title commands, author
%% commands, abstract environment, Computing Classification System
%% environment and commands, and keywords command.
\maketitle

\input{FreeMonadicProving}

% bib is defined in *.v-file
%\bibliography{paper}
\end{document}
